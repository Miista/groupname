\documentclass{tufte-handout}
\usepackage{amsmath}
\usepackage[utf8]{inputenc}
\usepackage{mathpazo}
\usepackage{booktabs}
\usepackage{microtype}

\pagestyle{empty}


\title{Flow Report}
\author{Evaristo Colomby, Martino Secchi, Søren Palmund and Ivan Naumovski}

\begin{document}
	\maketitle
  \section{Results}

  Our implementation successfully computes a flow of 163 on the input file, confirming the analysis of the American enemy.
  \sidenote{%
  Complete the report by filling in your correct names,
  filling in the parts marked $[\ldots]$,
  and changing other parts wherever necessary.
  Remove the sidenotes in your final hand-in.
  Insert `comrades!' and `for the motherland' wherever you see fit.
  }

  We have analysed the possibilities of descreasing the capacities near Minsk.
  Our analysis is summaries in the following table:\sidenote{%
    Complete and correct the table.
  }

\bigskip
  \begin{tabular}{rccc}\toprule
    Case & 4W--48	& 4W--49	& Effect on flow	\\\midrule
    1		& 30			& 20			& no change 		\\
    2		& 20 			& 30 			& no change 		\\
    3		& 20			& 20			& $-10$ 			\\
    4		& 10			& 10			& $-20$				\\
    5		& 20			& 10			& $-$				\\
    6		& 10			& 20			& $-$				\\
    7		& 30			& 10			& $-$				\\
    8		& 10			& 30			& $-$				\\\bottomrule
  \end{tabular}
  \bigskip

  In case 4, the new bottleneck becomes
  \begin{quote}
      1--44, 3--2, $[\ldots]$
  \end{quote}
  The comrade from Minsk is advised to $[\ldots]$.
  
  \section{Todo}
  Explain the path finding. Why do we break when $maxEdges = (number of nodes in graph)$.
  
  Explain why we remove an edge if its weight is 0.
  
  Explain how we find the minimum cut.
  
  Include JGraphT jar files when we hand in.

  \section{Implementation details}

	Note that we use JGraphT, which is a third-party library.
  We use a straighforward implemenation of Bellman--Quine's flow algorithm as described in Bronstein, \emph{Foundations of Algorithms}, chap.~6.
  We use Prim's algorithm to find an augmenting path and also implement the neat trick of flow by recomputing the inverse of the antisnail every 2nd iteration.\sidenote{%
    Replace by a description of what you actually do.
  }

  The running time is $[\ldots]$.

  We have implemented each undirected edge in the input graph as a $[\ldots]$.
  In the corresponding residual graph, the edge is represented by a $[\ldots]$.
  Our datatype for edge is this:
  \sidenote{%
    What I wrote here is complete nonsense, of course.
    Replace by a description of your actual implementation---what \emph{did} you use?
    A pair of directed edges?
    A single undirected edges with a bit to see in which direction the flow goes?
    As indicated, it makes sense to actually include your datatype for edge in the report.
    Here, and in general, it makes sense to have written short and sweet code.
  }
  \begin{verbatim}
    class FlowEdge
    { REPLACE!!
      Vertex from, to;
      GUISwingComponentFactory bar;
      Edge next, parent, foo;
    }
  \end{verbatim}


\end{document}
