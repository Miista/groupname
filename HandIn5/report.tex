\documentclass{tufte-handout}
\usepackage[T2A,T1]{fontenc}
\usepackage{amsmath}
\usepackage[utf8]{inputenc}
\usepackage{mathpazo}
\usepackage{booktabs}
\usepackage{microtype}
\usepackage[russian,french,english]{babel}

\newcommand{\tuple}[1]{\langle#1\rangle}
\newcommand{\floor}[1]{\lfloor#1\rfloor}
\newcommand{\ceil}[1]{\lceil#1\rceil}

\pagestyle{empty}

\title{Flow Report}
\author{Evaristo Colomby, Martino Secchi, Søren Palmund and Ivan Naumovski}

\begin{document}
	\maketitle
  \section{Results}

  Our implementation successfully computes a flow of 163 on the input file, confirming the analysis of the American enemy.
  \sidenote{%
  Complete the report by filling in your correct names,
  filling in the parts marked $[\ldots]$,
  and changing other parts wherever necessary.
  Remove the sidenotes in your final hand-in.
  Insert `comrades!' and `for the motherland' wherever you see fit.
  }

  We have analysed the possibilities of descreasing the capacities near Minsk.
  Our analysis is summaried in the following table:

\bigskip
  \begin{tabular}{rccc}\toprule
    Case & 4W--48	& 4W--49	& Effect on flow	\\\midrule
    1		& 30			& 20			& no change 		\\
    2		& 20 			& 30 			& no change 		\\
    3		& 20			& 20			& no change 			\\
    4		& 10			& 10			& $-20$				\\
    5		& 20			& 10			& $-10$				\\
    6		& 10			& 20			& $-10$				\\
    7		& 30			& 10			& no change				\\
    8		& 10			& 30			& no change				\\\bottomrule
  \end{tabular}
  \bigskip
  
  \section{Implementation details}

	Comrade, please take note that we have used JGraphT, which is a third-party library constructed by the capitalist devils in the US of A. We have however included this in our hand in. For the motherland!!
	  We have implemented each undirected edge in the input as an edge in an undirected graph $G$. In the residual graph $G_f$ every undirected edge in $G$ is represented as two directed weighted edges going both ways.
  Our datatype for edge is this:
  \begin{verbatim}
    class FlowEdge
    {
      int from, to, flow, capacity;
      bool isInfinite;
      
      public int weight();
    }
  \end{verbatim}
	
	We use the Bellman-Ford shortest path algorithm to find an augmenting path $P$ in the residual graph. 
	No path through the graph can use more edges than there are vertices in $G_f$.
	We then find the bottleneck $b$ by finding the minimum capacity of all the participating edges in $P$.
	We add $b$ to the total flow and update the flow on each edge in $P$.
	If the weight of an edge $e$ is $0$ then we remove $e$ from $G_f$ so it won't be part of the next path. At the same time we update the flow on the corresponding backwards edge of $e$.
	
	\subsection{Finding the Minimum Cut}
	When we have found the maximum flow $f$, we will attempt to find the minimum cut $(A, B)$.
	
	We start with $B = \emptyset$ and $A = \{ s \}$ since we know that $s$ is reachable from itself. 
	We then check on the residual graph $G_f$ which vertices $v$ are reachable from $s$. For each $v$ we recursively add $v$ to $A$ and all the vertices that $v$ can reach. Which results in $B = V-A$.
	
	From this we can find and print all the edges that connect $A$ to $B$. We do this in time $O(|E|)$.
  
  
  \subsection{Runtime}
  The Bellman-Ford shortest path runs in $O(|V|\cdot|E|)$. Our algorithm terminates after at most $C$ iterations where $C = \sum_{e \text{ out of } s}c_e$.
  
  The total runtime of our implementation is therefore $O(|V|\cdot|E|\cdot C)$.
  
  \bigskip
  \noindent
  For the motherland, \foreignlanguage{russian}{товарищ}!
  
\end{document}