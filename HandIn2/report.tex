\documentclass{tufte-handout}
\usepackage{amsmath}
\usepackage[utf8]{inputenc}
\usepackage{mathpazo}
\usepackage{booktabs}
\usepackage{microtype}

\newcommand{\tabitem}{~~\llap{\textbullet}~~}

\pagestyle{empty}

\title{Interval IP Report}
\author{Martino Secchi, Ivan Naumovski \& Søren Palmund}

\begin{document}
  \maketitle

  \section{Results}

  Our implementation produces the expected results on all
  input--output file pairs. Our algorithm uses the same number of kernels but some jobs are scheduled differently than what is in the output files. 

  \section{Implementation details}

  We have two classes that we use in the algorithm:
  \begin{itemize}
	\item \verb+Job+ which contains start time, end time, job id and kernel id. The latter two mainly for ordering and printing the output.
	\item \verb+Kernel+ is basically a wrapper around a \verb+Stack+ of \verb+Job+s. Each kernel has a unique id.
  \end{itemize}

  \noindent
  We store the kernels in a binary search tree indexed by what time they are available. This allows us to find an available kernel in $O(log n)$.\newline \noindent
  Each kernel is represented as a stack of jobs. This allows us to add a job to the kernel in $O(1)$.\linebreak
  
  \noindent
  For our stacks we use \verb+java.util.Stack+ and for our binary search tree we use \verb+java.util.TreeMap+.

\subsection{Run time}

  With these data structures, our implementation runs in time $O(n\ log\ n)$ on 
  inputs with $n$ jobs.

\subsection{Walkthrough}

  In the following we'll use $j$ to mean the current job, $j_s$ refers to the start time of $j$, and $j_f$ to mean the finish time of $j$.\linebreak

  We sort the input by increasing start time in time $O(n\ log\ n)$. Then for each job $j$ we...
  
  \begin{tabular}{| l | l |} \hline
    \textbf{Step} & \textbf{Run time} \\ \hline
    Attempt to find a kernel $k$ that is available at time $j_s$ or earlier & $O(log\ n)$ \\
    \tabitem If no kernel is available we'll create one and add $j$ to it &  \\
    \tabitem If one is available we'll simply add $j$ to $k$'s schedule &  \\
    We remove $k$ from the tree at $j_s$ and re-insert it at $j_f$ & $O(log\ n)$ \\
    \hline
  \end{tabular}
  
  Removing $k$ from the tree makes it unavailable at time $j_s$. Adding it again at time $j_f$ makes it available at that time. Doing this ensures that our tree is always sorted and up to date with available kernels.
  
\end{document}
