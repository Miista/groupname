\documentclass{tufte-handout}
\usepackage{amsmath}
\usepackage[utf8]{inputenc}
\usepackage{mathpazo}
\usepackage{booktabs}
\usepackage{microtype}

\pagestyle{empty}


\title{Closest Pair Report}
\author{Martino Secchi, Ivan Naumovski, and Søren Palmund}

\begin{document}
  \maketitle

  \section{Results}

  Our implementation produces the expected results on all input--output file pairs.

  The following table shows the closest pairs in some of the input files {\tt wc-instance-*.txt}.
  Here $n$ denotes the number of points in the input,
  and distance $\delta$.

  \bigskip\noindent
  \begin{tabular}{rr}\toprule
    $n$ & $\delta$ \\\midrule
    2 & 1.0 \\
    254 & 1.0 \\
    16382 & 1.0 \\
    65534 & 1.0 \\\bottomrule
  \end{tabular}


  \section{Implementation details}
  We have a global array of points called $points$ which contains all the points sorted by increasing $x$-value.

    Our running time is $O(n\log n)$ for $n$ points.
  \sidenote{Change or delete as necessary, add anything else you find interesting about your implemenation.}
    
  \subsection{Walkthrough}
  
  We call the method \verb+ClosestPair+ which takes a $start$ and $end$ which are indexes into the $points$ array.
  
  The method examines the number of points that are in between $start$ and $end$.\newline

  If there are only 3 points then we compute the closest pair.\newline
  
  If there are 2 points then we simply say that the two points are the closest pair.\newline
  
  In any other case we compute the median $m$ and split $points$ into two smaller subsets:\newline

  $S_1 = [start..m]$\newline
  $S_2 = ]m..end]$\newline

  We then call \verb+ClosestPair+ recursively to get the smallest distance on the left side $d_{left}$ and on the right side $d_{right}$.

We then set the delta $\delta = min(d_{left}, d_{right})$.\newline

Using $\delta$ we compute the range around $m$ (in the range $m \pm \delta$) which might hold pairs that are closer than what we've found so far.
To help with this we compute upper bound and lower bounds between which we'll consider every pair.

\end{document}
