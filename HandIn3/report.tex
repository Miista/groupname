\documentclass{tufte-handout}
\usepackage{amsmath}
\usepackage[utf8]{inputenc}
\usepackage{mathpazo}
\usepackage{booktabs}
\usepackage{microtype}

\newcommand{\tuple}[1]{\langle#1\rangle}
\newcommand{\floor}[1]{\lfloor#1\rfloor}
\newcommand{\ceil}[1]{\lceil#1\rceil}

\pagestyle{empty}


\title{Closest Pair Report}
\author{Martino Secchi, Ivan Naumovski, Evaristo Carvalho and Søren Palmund}

\begin{document}
\maketitle

  \section{Results}

  Our implementation produces the expected results on all input--output file pairs.

  The following table shows the closest pairs in some of the input files {\tt wc-instance-*.txt}.
  Here $n$ denotes the number of points in the input,
  and distance $\delta$.

  \bigskip\noindent
  \begin{tabular}{rr}\toprule
    $n$ & $\delta$ \\\midrule
    2 & 1.0 \\
    254 & 1.0 \\
    16382 & 1.0 \\
    65534 & 1.0 \\\bottomrule
  \end{tabular}


  \section{Implementation details}
    
  \subsection{Walkthrough}
  
  We call the method \verb+ClosestPair+ which takes as input a list of points sorted by X-values , called $input$. The above method returns the smallest distance between two points, named $delta$ and a y-sorted list of the inputted points in increasing order. \newline

Hence our method signature is as follows: $ClosestPair(points) \rightarrow \tuple{Double, points}$\newline
\noindent
 The method examines the number of points that are in the given $input$ list.\newline

We have three different cases:\newline \bigskip
\[
    opt(n) = 
    \begin{cases}
    	min(n_0, n_1, n_2)       			& \quad \text{if } |n| = 3\\
        min(n_0, n_1) 					& \quad \text{if } |n| = 2 \\
        opt( \ceil{n/2} ) + opt( \floor{n/2} )	& \quad \text{otherwise (default)} \\
    \end{cases}
\]\newline

\bigskip
\textbf{In-Depth of default case:}\newline
The split we do results in the two subsets, $m$ is the median of input\newline
  $Q = \lbrack 0 .. m \lbrack $\newline
  $R = \lbrack m .. input_{sz} \lbrack$ \newline 

These subsets are obtained by recursively calling \verb+ClosestPair+, the return type 
is a tuple which contains the smallest possible $delta$ and the Y-sorted $input$ for that recursive call.

We then set the delta $\delta = min(delta_{Q}, delta_{R})$.\newline

We then merge the two lists of Y-sorted points returned from recursive calls, namely the $Q_{ysort}$ and $R_{ysort}$ lists.

We allocate a new list $recombine$, and do a mergesort, hence\newline $recombine = Q_{y sort} \bigcup R_{y sort}$.
\newline

Using $\delta$ we compute the range on the X-axis around $m$ (in the range $m \pm \delta$) which 
might hold pairs that are closer than what we've found so far.
To help with this we compute upper bound and lower bounds on the X-values.\newline
For any points within these bounds in the $recombine$ list, we compare the distance between the point
 and the 15 following points.
If the distance is smaller than $\delta$ we update $\delta$ to be the smallest one.

\subsection{Run time}
By taking a closer look into our code, it seems as the majority of work is evenly divided throughout all the recursions.

We seem to spend linear time at each level during the merge of the sorted lists, and linear time to 
compare a given point to its 15 neighbours.
This is $15n$.\newline At the 4 bottom-most recursion levels we might compare less than 15.

Seeing the input is halved at every recursion we have a $\log n$ runtime, combined with the fact that we 
do linear amount of work for every recursion we expect our running time to be $O(n\log n)$ for $n$ points.

\end{document}